\section*{Introduction}
\phantomsection

A database index is a data structure that improves the speed of data retrieval operations on a database table at the cost of additional writes and storage space to maintain the index data structure. Indexes are used to quickly locate data without having to search every row in a database table every time a database table is accessed. Indexes can be created using one or more columns of a database table, providing the basis for both rapid random lookups and efficient access of ordered records.

An index is a copy of selected columns of data from a table that can be searched very efficiently that also includes a low-level disk block address or direct link to the complete row of data it was copied from. Some databases extend the power of indexing by letting developers create indexes on functions or expressions. 

Indexes are used in every database where performance matters. Although it has big benefit of performance improvement, we also have some consequences like:
\begin{itemize}
	\item More memory is used
	\item Add/Update/Delete operations on indexed columns, invokes index rebuild and makes this operations to be slower.
\end{itemize}

\clearpage